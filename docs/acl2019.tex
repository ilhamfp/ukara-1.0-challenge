%
% File acl2019.tex
%
%% Based on the style files for ACL 2018, NAACL 2018/19, which were
%% Based on the style files for ACL-2015, with some improvements
%%  taken from the NAACL-2016 style
%% Based on the style files for ACL-2014, which were, in turn,
%% based on ACL-2013, ACL-2012, ACL-2011, ACL-2010, ACL-IJCNLP-2009,
%% EACL-2009, IJCNLP-2008...
%% Based on the style files for EACL 2006 by 
%%e.agirre@ehu.es or Sergi.Balari@uab.es
%% and that of ACL 08 by Joakim Nivre and Noah Smith

\documentclass[11pt,a4paper]{article}
\usepackage[hyperref]{acl2019}
\usepackage{times}
\usepackage{latexsym}

\usepackage{url}

\aclfinalcopy % Uncomment this line for the final submission
%\def\aclpaperid{***} %  Enter the acl Paper ID here

%\setlength\titlebox{5cm}
% You can expand the titlebox if you need extra space
% to show all the authors. Please do not make the titlebox
% smaller than 5cm (the original size); we will check this
% in the camera-ready version and ask you to change it back.

\newcommand\BibTeX{B\textsc{ib}\TeX}

\title{Indonesian Essay Scoring using Bi-LSTM with Word Embedding Representation}

\author{Ilham Firdausi Putra \\
  Sekolah Teknik Elektro dan Informatika \\
  Institut Teknologi Bandung \\
  Bandung, Indonesia \\
  \texttt{ilhamfputra31@gmail.com} \\}


\date{}

\begin{document}
\maketitle
\begin{abstract}
  . The code and pretrained Word2vec word embedding will be made publicly available\footnote{\url{https://github.com/ilhamfp/ukara-1.0-challenge}}.
\end{abstract}

\section{Introduction}

Jelasin tentang essay scoring dan ukara challenge?

\section{Indonesian Essay Scoring}

Jelasin ngapain~\cite{Gusfield:97}



\subsection{Dataset}
dataset gmn isinya dan preprocessing yang dilakukan

\subsection{Word embedding}
gmn cara training bikin word embedidngnya pake Gensim~\cite{rehurek_lrec} dan data Opensubs bahasa Indonesia ~\cite{opensubs} kita se

\subsection{Bi-LSTM}
gmn detail modelnya

\subsection{Experiment}
gmn hasil experimentnya

\section{Conclusion}
Hehe selesai

\bibliography{acl2019}
\bibliographystyle{acl_natbib}

\appendix

\section{Hyperparameter Detail}
\subsection{Gensim Hyperparameter}
We use \verb|gensim.models.word2vec.Word2Vec| default parameter as of version 3.8.1.
\subsection{Model Hyperparameter}
We use the default parameter as of Keras version 2.3.0 and Tensorflow version 1.14.0 as the backend with the exception of the following:
\\

\textbf{Bi-LSTM}:
\begin{itemize}
\item \verb|units|: 50
\item \verb|return_sequences|: True
\item \verb|return_dropout|: 0.1
\item \verb|return_recurrent_dropout|: 0.1
\end{itemize}

\textbf{EarlyStopping}:
\begin{itemize}
\item \verb|monitor|: 'val\_f1'
\item \verb|min_delta|: 0.0001
\item \verb|patience|: 8
\item \verb|mode|: 'max'
\item \verb|baseline|: None
\item \verb|restore_best_weights|: True
\end{itemize}

\textbf{ReduceLROnPlateau}:
\begin{itemize}
\item \verb|monitor|: 'val\_f1'
\item \verb|factor|: 0.5
\item \verb|patience|: 3
\item \verb|mode|: 'max'
\item \verb|min_lr|: 1e-6
\end{itemize}

\end{document}
